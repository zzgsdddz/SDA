% Options for packages loaded elsewhere
\PassOptionsToPackage{unicode}{hyperref}
\PassOptionsToPackage{hyphens}{url}
%
\documentclass[
]{article}
\usepackage{amsmath,amssymb}
\usepackage{iftex}
\ifPDFTeX
  \usepackage[T1]{fontenc}
  \usepackage[utf8]{inputenc}
  \usepackage{textcomp} % provide euro and other symbols
\else % if luatex or xetex
  \usepackage{unicode-math} % this also loads fontspec
  \defaultfontfeatures{Scale=MatchLowercase}
  \defaultfontfeatures[\rmfamily]{Ligatures=TeX,Scale=1}
\fi
\usepackage{lmodern}
\ifPDFTeX\else
  % xetex/luatex font selection
\fi
% Use upquote if available, for straight quotes in verbatim environments
\IfFileExists{upquote.sty}{\usepackage{upquote}}{}
\IfFileExists{microtype.sty}{% use microtype if available
  \usepackage[]{microtype}
  \UseMicrotypeSet[protrusion]{basicmath} % disable protrusion for tt fonts
}{}
\makeatletter
\@ifundefined{KOMAClassName}{% if non-KOMA class
  \IfFileExists{parskip.sty}{%
    \usepackage{parskip}
  }{% else
    \setlength{\parindent}{0pt}
    \setlength{\parskip}{6pt plus 2pt minus 1pt}}
}{% if KOMA class
  \KOMAoptions{parskip=half}}
\makeatother
\usepackage{xcolor}
\usepackage[margin=1in]{geometry}
\usepackage{color}
\usepackage{fancyvrb}
\newcommand{\VerbBar}{|}
\newcommand{\VERB}{\Verb[commandchars=\\\{\}]}
\DefineVerbatimEnvironment{Highlighting}{Verbatim}{commandchars=\\\{\}}
% Add ',fontsize=\small' for more characters per line
\usepackage{framed}
\definecolor{shadecolor}{RGB}{248,248,248}
\newenvironment{Shaded}{\begin{snugshade}}{\end{snugshade}}
\newcommand{\AlertTok}[1]{\textcolor[rgb]{0.94,0.16,0.16}{#1}}
\newcommand{\AnnotationTok}[1]{\textcolor[rgb]{0.56,0.35,0.01}{\textbf{\textit{#1}}}}
\newcommand{\AttributeTok}[1]{\textcolor[rgb]{0.13,0.29,0.53}{#1}}
\newcommand{\BaseNTok}[1]{\textcolor[rgb]{0.00,0.00,0.81}{#1}}
\newcommand{\BuiltInTok}[1]{#1}
\newcommand{\CharTok}[1]{\textcolor[rgb]{0.31,0.60,0.02}{#1}}
\newcommand{\CommentTok}[1]{\textcolor[rgb]{0.56,0.35,0.01}{\textit{#1}}}
\newcommand{\CommentVarTok}[1]{\textcolor[rgb]{0.56,0.35,0.01}{\textbf{\textit{#1}}}}
\newcommand{\ConstantTok}[1]{\textcolor[rgb]{0.56,0.35,0.01}{#1}}
\newcommand{\ControlFlowTok}[1]{\textcolor[rgb]{0.13,0.29,0.53}{\textbf{#1}}}
\newcommand{\DataTypeTok}[1]{\textcolor[rgb]{0.13,0.29,0.53}{#1}}
\newcommand{\DecValTok}[1]{\textcolor[rgb]{0.00,0.00,0.81}{#1}}
\newcommand{\DocumentationTok}[1]{\textcolor[rgb]{0.56,0.35,0.01}{\textbf{\textit{#1}}}}
\newcommand{\ErrorTok}[1]{\textcolor[rgb]{0.64,0.00,0.00}{\textbf{#1}}}
\newcommand{\ExtensionTok}[1]{#1}
\newcommand{\FloatTok}[1]{\textcolor[rgb]{0.00,0.00,0.81}{#1}}
\newcommand{\FunctionTok}[1]{\textcolor[rgb]{0.13,0.29,0.53}{\textbf{#1}}}
\newcommand{\ImportTok}[1]{#1}
\newcommand{\InformationTok}[1]{\textcolor[rgb]{0.56,0.35,0.01}{\textbf{\textit{#1}}}}
\newcommand{\KeywordTok}[1]{\textcolor[rgb]{0.13,0.29,0.53}{\textbf{#1}}}
\newcommand{\NormalTok}[1]{#1}
\newcommand{\OperatorTok}[1]{\textcolor[rgb]{0.81,0.36,0.00}{\textbf{#1}}}
\newcommand{\OtherTok}[1]{\textcolor[rgb]{0.56,0.35,0.01}{#1}}
\newcommand{\PreprocessorTok}[1]{\textcolor[rgb]{0.56,0.35,0.01}{\textit{#1}}}
\newcommand{\RegionMarkerTok}[1]{#1}
\newcommand{\SpecialCharTok}[1]{\textcolor[rgb]{0.81,0.36,0.00}{\textbf{#1}}}
\newcommand{\SpecialStringTok}[1]{\textcolor[rgb]{0.31,0.60,0.02}{#1}}
\newcommand{\StringTok}[1]{\textcolor[rgb]{0.31,0.60,0.02}{#1}}
\newcommand{\VariableTok}[1]{\textcolor[rgb]{0.00,0.00,0.00}{#1}}
\newcommand{\VerbatimStringTok}[1]{\textcolor[rgb]{0.31,0.60,0.02}{#1}}
\newcommand{\WarningTok}[1]{\textcolor[rgb]{0.56,0.35,0.01}{\textbf{\textit{#1}}}}
\usepackage{graphicx}
\makeatletter
\def\maxwidth{\ifdim\Gin@nat@width>\linewidth\linewidth\else\Gin@nat@width\fi}
\def\maxheight{\ifdim\Gin@nat@height>\textheight\textheight\else\Gin@nat@height\fi}
\makeatother
% Scale images if necessary, so that they will not overflow the page
% margins by default, and it is still possible to overwrite the defaults
% using explicit options in \includegraphics[width, height, ...]{}
\setkeys{Gin}{width=\maxwidth,height=\maxheight,keepaspectratio}
% Set default figure placement to htbp
\makeatletter
\def\fps@figure{htbp}
\makeatother
\setlength{\emergencystretch}{3em} % prevent overfull lines
\providecommand{\tightlist}{%
  \setlength{\itemsep}{0pt}\setlength{\parskip}{0pt}}
\setcounter{secnumdepth}{-\maxdimen} % remove section numbering
\usepackage{fvextra}
\DefineVerbatimEnvironment{Highlighting}{Verbatim}{breaklines,commandchars=\\\{\}}
\ifLuaTeX
  \usepackage{selnolig}  % disable illegal ligatures
\fi
\usepackage{bookmark}
\IfFileExists{xurl.sty}{\usepackage{xurl}}{} % add URL line breaks if available
\urlstyle{same}
\hypersetup{
  pdftitle={SDA Group Submission Assignment Assign3},
  pdfauthor={MengliFeng (2720589) and PepijnVanOostveen (2801582)},
  hidelinks,
  pdfcreator={LaTeX via pandoc}}

\title{SDA Group Submission Assignment Assign3}
\usepackage{etoolbox}
\makeatletter
\providecommand{\subtitle}[1]{% add subtitle to \maketitle
  \apptocmd{\@title}{\par {\large #1 \par}}{}{}
}
\makeatother
\subtitle{Group Gr18}
\author{MengliFeng (2720589) and PepijnVanOostveen (2801582)}
\date{}

\begin{document}
\maketitle

\section{Exercise 1}\label{exercise-1}

\subsection{a.}\label{a.}

\begin{Shaded}
\begin{Highlighting}[]
\CommentTok{\# Set seed for reproducibility}
\FunctionTok{set.seed}\NormalTok{(}\DecValTok{123}\NormalTok{)}

\CommentTok{\# Generate random sample from t{-}distribution with 3 degrees of freedom}
\NormalTok{n }\OtherTok{\textless{}{-}} \DecValTok{20}
\NormalTok{sample\_data }\OtherTok{\textless{}{-}} \FunctionTok{rt}\NormalTok{(n, }\AttributeTok{df =} \DecValTok{3}\NormalTok{)}

\CommentTok{\# Define different kernel types and colors}
\NormalTok{kernels }\OtherTok{\textless{}{-}} \FunctionTok{c}\NormalTok{(}\StringTok{"gaussian"}\NormalTok{, }\StringTok{"epanechnikov"}\NormalTok{, }\StringTok{"rectangular"}\NormalTok{, }\StringTok{"triangular"}\NormalTok{)}
\NormalTok{colors\_kernels }\OtherTok{\textless{}{-}} \FunctionTok{c}\NormalTok{(}\StringTok{"red"}\NormalTok{, }\StringTok{"blue"}\NormalTok{, }\StringTok{"green"}\NormalTok{, }\StringTok{"purple"}\NormalTok{)}

\CommentTok{\# Define different bandwidth choices and colors}
\NormalTok{bandwidths }\OtherTok{\textless{}{-}} \FunctionTok{c}\NormalTok{(}\FunctionTok{density}\NormalTok{(sample\_data)}\SpecialCharTok{$}\NormalTok{bw, }\FloatTok{0.3}\NormalTok{, }\FloatTok{1.5}\NormalTok{)}
\NormalTok{colors\_bandwidths }\OtherTok{\textless{}{-}} \FunctionTok{c}\NormalTok{(}\StringTok{"red"}\NormalTok{, }\StringTok{"blue"}\NormalTok{, }\StringTok{"green"}\NormalTok{)}

\CommentTok{\# Adjust plot margins to make space for legends}
\FunctionTok{par}\NormalTok{(}\AttributeTok{mfrow =} \FunctionTok{c}\NormalTok{(}\DecValTok{1}\NormalTok{, }\DecValTok{2}\NormalTok{), }\AttributeTok{mar =} \FunctionTok{c}\NormalTok{(}\DecValTok{5}\NormalTok{, }\DecValTok{4}\NormalTok{, }\DecValTok{6}\NormalTok{, }\DecValTok{4}\NormalTok{))  }\CommentTok{\# Extra right margin for the legend}

\CommentTok{\# Plot histogram with different kernel choices}
\FunctionTok{hist}\NormalTok{(sample\_data, }\AttributeTok{probability =} \ConstantTok{TRUE}\NormalTok{, }\AttributeTok{main =} \StringTok{"Kernels"}\NormalTok{, }\AttributeTok{col =} \StringTok{"lightgray"}\NormalTok{, }\AttributeTok{border =} \StringTok{"black"}\NormalTok{)}

\ControlFlowTok{for}\NormalTok{ (i }\ControlFlowTok{in} \FunctionTok{seq\_along}\NormalTok{(kernels)) \{}
  \FunctionTok{lines}\NormalTok{(}\FunctionTok{density}\NormalTok{(sample\_data, }\AttributeTok{kernel =}\NormalTok{ kernels[i]), }\AttributeTok{col =}\NormalTok{ colors\_kernels[i], }\AttributeTok{lwd =} \DecValTok{2}\NormalTok{)}
\NormalTok{\}}

\CommentTok{\# Add legend outside the plot}
\FunctionTok{legend}\NormalTok{(}\StringTok{"topright"}\NormalTok{, }\AttributeTok{inset =} \FunctionTok{c}\NormalTok{(}\SpecialCharTok{{-}}\FloatTok{0.3}\NormalTok{, }\DecValTok{0}\NormalTok{), }\AttributeTok{legend =}\NormalTok{ kernels, }\AttributeTok{col =}\NormalTok{ colors\_kernels, }\AttributeTok{lwd =} \DecValTok{2}\NormalTok{, }\AttributeTok{cex =} \FloatTok{0.5}\NormalTok{, }\AttributeTok{title =} \StringTok{"Kernels"}\NormalTok{, }\AttributeTok{xpd =} \ConstantTok{TRUE}\NormalTok{)}

\CommentTok{\# Plot histogram with different bandwidth choices}
\FunctionTok{hist}\NormalTok{(sample\_data, }\AttributeTok{probability =} \ConstantTok{TRUE}\NormalTok{, }\AttributeTok{main =} \StringTok{"Bandwidths (Gaussian)"}\NormalTok{, }\AttributeTok{col =} \StringTok{"lightgray"}\NormalTok{, }\AttributeTok{border =} \StringTok{"black"}\NormalTok{)}

\ControlFlowTok{for}\NormalTok{ (i }\ControlFlowTok{in} \FunctionTok{seq\_along}\NormalTok{(bandwidths)) \{}
  \FunctionTok{lines}\NormalTok{(}\FunctionTok{density}\NormalTok{(sample\_data, }\AttributeTok{bw =}\NormalTok{ bandwidths[i]), }\AttributeTok{col =}\NormalTok{ colors\_bandwidths[i], }\AttributeTok{lwd =} \DecValTok{2}\NormalTok{)}
\NormalTok{\}}

\CommentTok{\# Add legend outside the plot}
\FunctionTok{legend}\NormalTok{(}\StringTok{"topright"}\NormalTok{, }\AttributeTok{inset =} \FunctionTok{c}\NormalTok{(}\SpecialCharTok{{-}}\FloatTok{0.3}\NormalTok{, }\DecValTok{0}\NormalTok{), }\AttributeTok{legend =} \FunctionTok{paste}\NormalTok{(}\StringTok{"bw ="}\NormalTok{, }\FunctionTok{round}\NormalTok{(bandwidths, }\DecValTok{2}\NormalTok{)), }\AttributeTok{col =}\NormalTok{ colors\_bandwidths, }\AttributeTok{lwd =} \DecValTok{2}\NormalTok{, }\AttributeTok{cex =} \FloatTok{0.5}\NormalTok{,}
       \AttributeTok{title =} \StringTok{"Bandwidths"}\NormalTok{, }\AttributeTok{xpd =} \ConstantTok{TRUE}\NormalTok{)}

\CommentTok{\# {-}{-}{-} Add an Overall Title {-}{-}{-}}
\FunctionTok{mtext}\NormalTok{(}\StringTok{"Density Estimation with different kernels and kernel bandwidths"}\NormalTok{,}\AttributeTok{line =} \DecValTok{4}\NormalTok{, }\AttributeTok{cex =} \DecValTok{1}\NormalTok{, }\AttributeTok{font =} \DecValTok{1}\NormalTok{, }\AttributeTok{adj =} \DecValTok{1}\NormalTok{)}
\end{Highlighting}
\end{Shaded}

\includegraphics{SDA_submission_template_files/figure-latex/unnamed-chunk-1-1.pdf}

\subsection{b.}\label{b.}

From the generated plots, we can observe: • Effect of Kernel Choice:
Different kernels produce similar overall shapes, but their smoothness
varies slightly. The Gaussian kernel is the smoothest, while the
rectangular kernel has more abrupt changes. • Effect of Bandwidth
Choice: The bandwidth has a much larger influence than the kernel. A
smaller bandwidth (0.3) captures more fluctuations in the data, while a
larger bandwidth (1.5) smooths out more features. • Key Influence:
Bandwidth choice has a bigger impact on the estimator compared to kernel
choice.

\subsection{c.}\label{c.}

\begin{Shaded}
\begin{Highlighting}[]
\NormalTok{h\_opt }\OtherTok{\textless{}{-}} \ControlFlowTok{function}\NormalTok{(x) \{}
\NormalTok{  sigma\_hat }\OtherTok{\textless{}{-}} \FunctionTok{min}\NormalTok{(}\FunctionTok{sd}\NormalTok{(x), }\FunctionTok{IQR}\NormalTok{(x) }\SpecialCharTok{/} \FloatTok{1.34}\NormalTok{)  }\CommentTok{\# Compute standard deviation and interquartile range}
\NormalTok{  h\_optimal }\OtherTok{\textless{}{-}} \FloatTok{1.06} \SpecialCharTok{*}\NormalTok{ sigma\_hat }\SpecialCharTok{*} \FunctionTok{length}\NormalTok{(x)}\SpecialCharTok{\^{}}\NormalTok{(}\SpecialCharTok{{-}}\DecValTok{1}\SpecialCharTok{/}\DecValTok{5}\NormalTok{)  }\CommentTok{\# Optimal bandwidth formula}
  \FunctionTok{return}\NormalTok{(h\_optimal)}
\NormalTok{\}}

\CommentTok{\# Compute optimal bandwidth for the sample}
\NormalTok{h\_opt\_value }\OtherTok{\textless{}{-}} \FunctionTok{h\_opt}\NormalTok{(sample\_data)}

\CommentTok{\# Compare with R\textquotesingle{}s default bandwidth}
\NormalTok{default\_bw }\OtherTok{\textless{}{-}} \FunctionTok{density}\NormalTok{(sample\_data)}\SpecialCharTok{$}\NormalTok{bw}

\CommentTok{\# Print results}
\FunctionTok{cat}\NormalTok{(}\StringTok{"Optimal Bandwidth (h\_opt):"}\NormalTok{, h\_opt\_value, }\StringTok{"}\SpecialCharTok{\textbackslash{}n}\StringTok{"}\NormalTok{)}
\end{Highlighting}
\end{Shaded}

\begin{verbatim}
## Optimal Bandwidth (h_opt): 0.831087
\end{verbatim}

\begin{Shaded}
\begin{Highlighting}[]
\FunctionTok{cat}\NormalTok{(}\StringTok{"Default R Bandwidth:"}\NormalTok{, default\_bw, }\StringTok{"}\SpecialCharTok{\textbackslash{}n}\StringTok{"}\NormalTok{)}
\end{Highlighting}
\end{Shaded}

\begin{verbatim}
## Default R Bandwidth: 0.7056399
\end{verbatim}

\section{Exercise 2}\label{exercise-2}

\subsection{a.}\label{a.-1}

I expect that \(f(t)=0\) only for \(t<0\). As \(f\) is only \(0\) when
there is no chance that you wait for that amount of time and you can't
wait for a negative amount of time. For any other amount of time there
is a non-zero chance as you can wait anywhere from 0 minutes (someone
else called the elevator at the right time that you can enter the
elevator with them) and a practically infinite amount of time (the
elevator is out of order).

\subsection{b.}\label{b.-1}

\begin{Shaded}
\begin{Highlighting}[]
\CommentTok{\# load the dataset}
\NormalTok{wait\_times }\OtherTok{\textless{}{-}} \FunctionTok{readRDS}\NormalTok{(}\StringTok{"waiting\_times.RDS"}\NormalTok{)}

\CommentTok{\# log{-}transform as suggested by the hint}
\NormalTok{log\_times }\OtherTok{=} \FunctionTok{log}\NormalTok{(wait\_times)}

\CommentTok{\# estimating the density of the transformed sample}
\NormalTok{y\_seq }\OtherTok{\textless{}{-}} \FunctionTok{seq}\NormalTok{(}\FunctionTok{min}\NormalTok{(log\_times), }\FunctionTok{max}\NormalTok{(log\_times), }\AttributeTok{length.out =} \DecValTok{512}\NormalTok{)}
\NormalTok{density\_y }\OtherTok{\textless{}{-}} \FunctionTok{density}\NormalTok{(log\_times, }\AttributeTok{from =} \FunctionTok{min}\NormalTok{(y\_seq), }\AttributeTok{to =} \FunctionTok{max}\NormalTok{(y\_seq))}

\CommentTok{\# Transform back to the original scale}
\NormalTok{x\_seq }\OtherTok{\textless{}{-}} \FunctionTok{exp}\NormalTok{(y\_seq)}
\NormalTok{density\_x }\OtherTok{\textless{}{-}}\NormalTok{ density\_y}\SpecialCharTok{$}\NormalTok{y }\SpecialCharTok{/} \FunctionTok{exp}\NormalTok{(y\_seq)}

\CommentTok{\# plot histogram}
\FunctionTok{hist}\NormalTok{(wait\_times, }\AttributeTok{probability =} \ConstantTok{TRUE}\NormalTok{, }\AttributeTok{breaks =} \DecValTok{30}\NormalTok{, }\AttributeTok{main =} \StringTok{"Kernel Density Estimation"}\NormalTok{, }\AttributeTok{xlab =} \StringTok{"Waiting Time (minutes)"}\NormalTok{)}

\CommentTok{\# Add KDE line}
\FunctionTok{lines}\NormalTok{(x\_seq, density\_x, }\AttributeTok{col =} \StringTok{"red"}\NormalTok{,}\AttributeTok{lwd =} \DecValTok{2}\NormalTok{)}
\end{Highlighting}
\end{Shaded}

\includegraphics{SDA_submission_template_files/figure-latex/unnamed-chunk-3-1.pdf}

\subsection{c.}\label{c.-1}

\begin{Shaded}
\begin{Highlighting}[]
\FunctionTok{library}\NormalTok{(EnvStats)}
\end{Highlighting}
\end{Shaded}

\begin{verbatim}
## 
## Attaching package: 'EnvStats'
\end{verbatim}

\begin{verbatim}
## The following objects are masked from 'package:stats':
## 
##     predict, predict.lm
\end{verbatim}

\begin{Shaded}
\begin{Highlighting}[]
\CommentTok{\# QQ{-}Plot for Exponential distribution}
\FunctionTok{qqPlot}\NormalTok{(wait\_times, }\AttributeTok{distribution =} \StringTok{"exp"}\NormalTok{, }\AttributeTok{estimate.params =} \ConstantTok{TRUE}\NormalTok{, }\AttributeTok{add.line =} \ConstantTok{TRUE}\NormalTok{, }\AttributeTok{line.col =} \StringTok{"red"}\NormalTok{, }\AttributeTok{main =} \StringTok{"QQ{-}Plot: Sample vs. Exponential"}\NormalTok{)}
\end{Highlighting}
\end{Shaded}

\includegraphics{SDA_submission_template_files/figure-latex/unnamed-chunk-4-1.pdf}
The points seem to follow the line pretty well so I think that an
exponential distribution is an appropriate choice to model the sample.

\subsection{d.}\label{d.}

\begin{Shaded}
\begin{Highlighting}[]
\CommentTok{\# We know that the rate MLE for the rate lambda is 1 divided by the sample mean (the waiting times)}
\NormalTok{l\_mle }\OtherTok{\textless{}{-}} \DecValTok{1} \SpecialCharTok{/} \FunctionTok{mean}\NormalTok{(wait\_times)}

\NormalTok{x\_seq }\OtherTok{\textless{}{-}} \FunctionTok{seq}\NormalTok{(}\FunctionTok{min}\NormalTok{(wait\_times), }\FunctionTok{max}\NormalTok{(wait\_times), }\AttributeTok{length.out =} \DecValTok{512}\NormalTok{)}

\CommentTok{\# Now we make make the theoretical density function using our estimate}
\NormalTok{exp\_density }\OtherTok{\textless{}{-}}\NormalTok{ l\_mle }\SpecialCharTok{*} \FunctionTok{exp}\NormalTok{(}\SpecialCharTok{{-}}\NormalTok{l\_mle }\SpecialCharTok{*}\NormalTok{ x\_seq)}

\CommentTok{\# Plot the histogram again}
\FunctionTok{hist}\NormalTok{(wait\_times, }\AttributeTok{probability =} \ConstantTok{TRUE}\NormalTok{, }\AttributeTok{breaks =} \DecValTok{30}\NormalTok{, }\AttributeTok{main =} \StringTok{"Hist with Exponential MLE Density"}\NormalTok{, }\AttributeTok{xlab =} \StringTok{"Waiting Time (minutes)"}\NormalTok{)}

\CommentTok{\# Add a line for our theoretical density function}
\FunctionTok{lines}\NormalTok{(x\_seq, exp\_density, }\AttributeTok{col =} \StringTok{"red"}\NormalTok{, }\AttributeTok{lwd =} \DecValTok{2}\NormalTok{)}
\FunctionTok{legend}\NormalTok{(}\StringTok{"topright"}\NormalTok{, }\AttributeTok{legend =} \FunctionTok{c}\NormalTok{(}\StringTok{"Exponential MLE Density"}\NormalTok{), }\AttributeTok{col =} \StringTok{"red"}\NormalTok{, }\AttributeTok{lwd =} \DecValTok{2}\NormalTok{)}
\end{Highlighting}
\end{Shaded}

\includegraphics{SDA_submission_template_files/figure-latex/unnamed-chunk-5-1.pdf}
The exponential density is worse than the estimator found b since the
sample doesn't exactly follow an exponential distribution. Instead the
sample follows an exponential distribution, but with with an added peak
close to zero because relatively often you don't have to wait for the
elevator at all because someone else has already called it.

\end{document}
